\documentclass[a4paper,10pt]{article}

\usepackage{hyperref}
\usepackage[dutch]{babel}

\hypersetup{colorlinks=true, linkcolor=cyan}

%opening
\title{Plan van Aanpak - Gedistribueerde systemen 2010}
\author{Team 7 \\
Jan Laan - 5756529\\
Bas Vlaszaty - 5783445 \\
Saidou Diallo - 6215297\\
Koos van Strien - }

\begin{document}

  \maketitle
  
  \section{Inleiding}
  Voor het college Gedistribueerde Systemem 2010 maken wij een chatserver. Deze server is onderdeel van een groep servers die samen een chat-netwerk vormen. De server moet rekening houden met meerdere andere servers en meerdere clients die eraan verbonden zijn. Ook heeft de server een continue verbinding met de \textit{control server}.
  
  \section{Implementatie}
  Als implementatietaal is gekozen voor Python. We zijn niet heel bekend met deze taal. Sommigen hebben wel wat in Python gedaan, sommigen helemaal niet. Dit is een mooie gelegenheid om een nieuwe taal te leren kennen. Daarnaast is Python een taal waarin je relatief snel kunt programmeren. Het is compact en zit vol met handige functies. Omdat het een ge\"interpreteerde taal is is het niet heel snel. Wij denken dat dit geen problemen gaat opleveren omdat de server niet zwaar belast gaat worden. Als dit wel het geval zou zijn is het handiger om voor een gecompileerde taal te kiezen.

  \section{Analyse}

  \section{Software}

  \section{Werkverdeling}

  \section{Tijdsplanning}
  \begin{itemize}
    \item \textbf{Week 1 (31 Maart)}: Plan van aanpak, werkverdeling
    \item \textbf{Week 2 (7 April)}: Programmeren
    \item \textbf{Week 3 (14 April)}: Programmeren
    \item \textbf{Week 4 (21 April)}: Werkende versie af
    \item \textbf{Week 5 (28 April)}: Stabiele versie af
    \item \textbf{Week 6 (10 Mei)}: laatste tests, afronden verslag
    \item \textbf{Week 7 (12 Mei)}: gezamenlijke test

  \end{itemize}


\end{document}
